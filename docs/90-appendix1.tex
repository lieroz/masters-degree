\chapter{Приложение А}
\label{cha:appendix1}

%\begin{figure}
%\centering
%\caption{Картинка в приложении. Страшная и ужасная.}
%\end{figure}

%%% Local Variables: 
%%% mode: latex
%%% TeX-master: "rpz"
%%% End: 
%,caption={Код создания модели Word2Vec},
\captionsetup{justification=centering}

\begin{lstlisting}[language=c++,caption={Код функуции выделения памяти},numbers=none,xleftmargin=.1\textwidth,xrightmargin=.1\textwidth]
void *malloc(size_t size)
{
	void *ptr{nullptr};
	if (size <= smallObjectsSizeLimit)
	{
		size = roundUpToNextPowerOf2(size);
		auto &registry = getSmallObjectsRegistry(size);
		assert(registry.pop(ptr));
		
		uint64_t controlBits = (size << highestVirtualSpaceBit) | smallObjectMask;
		ptr = reinterpret_cast<void *>(reinterpret_cast<uint64_t>(ptr) | controlBits);
	}
	else
	{
		size = getNextNearestDivisibleByPageSize(size);
		
		if (size <= largeObjectsSizeLimit)
		{
			static constexpr uint64_t hardLimit = ((1 << 15) - 1) * pageSize;
			assert(size <= hardLimit);
			
			auto &registry = getLargeObjectsRegistry(size);
			registry.pop(ptr);
			
			uint64_t controlBits = (size / pageSize) << highestVirtualSpaceBit;
			ptr = reinterpret_cast<void *>(reinterpret_cast<uint64_t>(ptr) | controlBits);
		}
		else
		{
			ptr = mmapImpl(pageSize + size);
			*static_cast<uint64_t *>(ptr) = size;
			ptr = static_cast<std::byte *>(ptr) + pageSize;
		}
	}
	return ptr;
}
\end{lstlisting}