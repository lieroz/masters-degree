\chapter{Технологический раздел}
\section{Выбор средств программной реализации}
ПО разработано под ОС Linux и написано на языке С++17. Исходные коды ОС Linux находятся в открытом доступе, благодаря этому написано много статей и книг про ее внутреннее устройство, что облегчает поиск информации об ОС. Так же имеется документация по всем системным вызовам и можно обратиться к исходным кодам, если предоставленной информации недостаточно. Выбор языка обусловлен тем, что необходима низкоуровневая работа с памятью и API ОС. Язык С++ является надстройкой над языком С и поэтому может использовать все его встроенные методы для работы с системными вызовами. Помимо этого в языке реализованы такие парадигмы как объектно-ориентированное и функциональное программирование и имеется шаблонизация функций и классов для обобщения кода.

Для написания кода испольpовался редактор vim, это очень простой консольный редактор, для которого существует огромное количество плагинов. Для данной работы был подобран свой список плагинов и редактор был настроен исходя из предпочтений автора. При отладке программы использовался дебаггер GDB. Данный дебаггер является стандартным под ОС Linux и имеет хорошую документацию и сообщество, которое помогает в его освоении. Имеется поддержка написания собственных функций. Дебаггер выводит строку, на которой проихошла ошибка и список фреймов стека, по которым можно перейти и распечатать всю нужныю информацию.

\section{Требования к аппаратному и программному обеспечению}
ПО реализовано под архитектуру x86\_64 и использует 16 верхних бит из 64-х для хранения метаданных. Для коректной его работы необходимо отключить PAE (если включено) при загрузке ОС. Перед тем как использовать библиотеку необходимо ее подгрузить в переменные среды линковщика через LD\_PRELOAD: \textbf{export LD\_PRELOAD=/path/to/library.so}.

\section{Входные и выходные параметры ПО}
На вход подается рамер блока, который необходимо выделить. Размер может быть любым, если запрошено памяти больше чем есть в системе или метод отображения памяти вернул ошибку, аллокатор запишет ошибку в станрадтный поток вывода и завершит работу процесса. Выходным параметром является адрес, который содержит в себе метаданные об объекте.

\section{Структура разработанного ПО}
% CODE GOES HERE

\section{Вывод}