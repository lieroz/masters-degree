\chapter{Технологический раздел}
\section{Выбор средств программной реализации}
ПО разработано под ОС Linux и написано на языке С++17. Исходные коды ОС Linux находятся в открытом доступе, благодаря этому написано много статей и книг про ее внутреннее устройство, что облегчает поиск информации об ОС. Так же имеется документация по всем системным вызовам и можно обратиться к исходным кодам, если предоставленной информации недостаточно. Выбор языка обусловлен тем, что необходима низкоуровневая работа с памятью и API ОС. Язык С++ является надстройкой над языком С и поэтому может использовать все его встроенные методы для работы с системными вызовами. Помимо этого в языке реализованы такие парадигмы как объектно-ориентированное и функциональное программирование и имеется шаблонизация функций и классов для обобщения кода.

Для написания кода испольpовался редактор vim, это очень простой консольный редактор, для которого существует огромное количество плагинов. Для данной работы был подобран свой список плагинов и редактор был настроен исходя из предпочтений автора. При отладке программы использовался дебаггер GDB. Данный дебаггер является стандартным под ОС Linux и имеет хорошую документацию и сообщество, которое помогает в его освоении. Имеется поддержка написания собственных функций. Дебаггер выводит строку, на которой проихошла ошибка и список фреймов стека, по которым можно перейти и распечатать всю нужныю информацию.

\section{Входные и выходные параметры ПО}
На вход подается рамер блока, который необходимо выделить. Размер может быть любым, если запрошено памяти больше чем есть в системе или метод отображения памяти вернул ошибку, аллокатор запишет ошибку в станрадтный поток вывода и завершит работу процесса. Выходным параметром является адрес, который содержит в себе метаданные об объекте.

ПО реализовано под архитектуру x86\_64 и использует 16 верхних бит из 64-х для хранения метаданных. Для коректной его работы необходимо отключить PAE (если включено) при загрузке ОС. Перед тем как использовать библиотеку необходимо ее подгрузить в переменные среды линковщика через LD\_PRELOAD: \textbf{export LD\_PRELOAD=/path/to/library.so}.

\section{Рекомендации к использованию ПО}
Для разных типов объектов реализуются свои классы управления. Все операции с возвращаемой памятью должны проводиться через функцию \textit{getWorkingAddress}, которая имеет следующий вид:

\begin{lstlisting}[language=c++,numbers=none]
// 0000000000000000 -> 00007fffffffffff => canonical low address space half
// 00007fffffffffff -> ffff800000000000 => illegal addresses, unused
// ffff800000000000 -> ffffffffffffffff => canonical high address space half
template<typename T>
T *getWorkingAddress(T *value)
{
	static constexpr uint64_t workingAddressMask =
		(static_cast<uint64_t>(1) << highestVirtualSpaceBit) - 1;
	uint64_t address = reinterpret_cast<uint64_t>(value) & workingAddressMask;
	
	if ((reinterpret_cast<uint64_t>(value) &
		(static_cast<uint64_t>(1) << (highestVirtualSpaceBit - 1))) != 0)
	{
		static constexpr uint64_t highHalf = static_cast<uint64_t>(0x1ffff)
			<< (highestVirtualSpaceBit - 1);
		address |= highHalf;
	}
	
	return reinterpret_cast<T *>(address);
}
\end{lstlisting}

Запрос на выделение использует функцию \textit{malloc} данной библиотеки:
\begin{lstlisting}[language=c++,numbers=none]
void *malloc(size_t size)
{
	void *ptr{nullptr};
	if (size <= smallObjectsSizeLimit)
	{
		size = roundUpToNextPowerOf2(size);
		auto &registry = getSmallObjectsRegistry(size);
		assert(registry.pop(ptr));
		
		uint64_t controlBits = (size << highestVirtualSpaceBit) | smallObjectMask;
		ptr = reinterpret_cast<void *>(reinterpret_cast<uint64_t>(ptr) | controlBits);
	}
	else
	{
		size = getNextNearestDivisibleByPageSize(size);
		
		if (size <= largeObjectsSizeLimit)
		{
			static constexpr uint64_t hardLimit = ((1 << 15) - 1) * pageSize;
			assert(size <= hardLimit);
			
			auto &registry = getLargeObjectsRegistry(size);
			registry.pop(ptr);
			
			uint64_t controlBits = (size / pageSize) << highestVirtualSpaceBit;
			ptr = reinterpret_cast<void *>(reinterpret_cast<uint64_t>(ptr) | controlBits);
		}
		else
		{
			ptr = mmapImpl(pageSize + size);
			*static_cast<uint64_t *>(ptr) = size;
			ptr = static_cast<std::byte *>(ptr) + pageSize;
		}
	}
	return ptr;
}
\end{lstlisting}

Для освобождения блока памяти необходимо использовать функцию \textit{free}:
\begin{lstlisting}[language=c++,numbers=none]
void free(void *ptr)
{
	uint64_t address = reinterpret_cast<uint64_t>(ptr);
	size_t sizeClass = (address >> highestVirtualSpaceBit) & 0x7fff;
	
	ptr = getWorkingAddress(ptr);
	
	if (sizeClass <= smallObjectsSizeLimit)
	{
		auto &registry = getSmallObjectsRegistry(sizeClass);
		registry.push(ptr);
	}
	else if (sizeClass != 0)
	{
		auto &registry = getLargeObjectsRegistry(sizeClass * pageSize);
		registry.push(ptr);
	}
	else
	{
		ptr = static_cast<std::byte *>(ptr) - pageSize;
		munmapImpl(ptr, *static_cast<uint64_t *>(ptr));
	}
}
\end{lstlisting}