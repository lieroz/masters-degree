\chapter{Технологический раздел}
\section{Выбор средств разработки}
Программный продукт разработан специально под ОС Linux и может работать только в 64-х битной ее версии, при этом адресация должна включать в себя только первые 48 бит, тоесть должно быть отключено PAE. Данный выбор обусловлен открытостью ОС Linux, ее хорошей документацией и большим списком доступной литературы по архитектуре самой ОС. Данная ОС сильно распространена как среди пользователей, так и в серверном сегменте, поэтому имеется широкая возможность выбора языка программирования и сопутствующих инструментов разработки.

\subsection{Выбор языка программирования}
Для написания кода библиотеки был выбран язык С++ и был использован его 17-й стандарт. Язык позволяет работать на достаточно низком уровне, при этом предоставляет возможность описывать все высокоуровневыми конструкциями. Он сразу интегрируется с языком С, на котором и написана ОС Linux, и можно испольщовать все предоставляемые системой системный вызовы. Преимущества данного языка:
\begin{itemize}
	\item возможность низкоуровневого программирования; имеется интеграция с языком С и можно делать ассемблерные вставки прямо в коде;
	\item язык развивается с 80-х годов прошлого века, поэтому огромное количество компиляторов его реализуют;
	\item скорость работы;
	\item ручное управление памятью, самый выжный аспект для данной работы, потому что можно на системном уровне реализовать свое управление оперативной памятью и не придется пресобирать всю среду языка для проверки работы библиотеки.
\end{itemize}

\subsection{Выбор среды разработки и отладки}
Для написания кода испольщовался редактор vim, это очень простой консольный редактор, для которого существует огромное количество плагинов. Для данной работы был подобран свой список плагинов и редактор был настроен исходя из предпочтений автора.

При отладке программы использовался дебаггер GDB. Данный дебаггер является стандартным под ОС Linux и имеет хорошую документацию и сообщество, которое помогает в его освоении. Имеется поддержка написания собственных функций. Дебаггер выводит строку, на которой проихошла ошибка и список фреймов стека, по которым можно перейти и распечатать всю нужныю информацию.

\section{Система контроля версий}
Весь код хранился в системе контроля версий Git. Она хранит всю историю изменений и в любой момент можно вернуться на нужную версию, чтобы перенести код или наоборот удалить все новые изменения. Также система решает вопрос связанный с резервным копированием, ее особенности:
\begin{itemize}
	\item предоставляет широкий набор инструментов по управлению изменениями и историей изменений;
	\item входит в станлдартные пакеты всех дистрибутивов Linux;
	\item поддерживается такими хостингами как GitHub/GitLab/BitBucket.
\end{itemize}