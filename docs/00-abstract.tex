% Также можно использовать \Referat, как в оригинале
\begin{abstract}
Расчетно-пояснительная записка 69 с., 4 р., 39 рис., 6 табл., 23 источника.

Объектом разработки является приложение, реализующее метод метод распределения памяти.

Цель работы — спроектировать и реализовать метод распределения оперативной памяти для многопоточных приложений под архитектуру х86\_86.

В рамках работы решены следующие задачи:
\begin{itemize}
	\item проведен анализ существующих методов выделения виртуальной памяти приложениям;
	\item проведён анализ существующих способов выделения физической памяти приложениям;
	\item разработан метод выделения памяти, сокращающий накладные расходы при выделении памяти;
	\item разработан алгоритм распределения памяти и структуа ПО;
	\item исследованы характеристики предложенного метода в сравнении с прототипом.
\end{itemize}

Практическая область применения — распределение памяти для приложений пользователей написанных под ОС Linux.

В аналитическом разделе работы приведён анализ известных методов решения задачи распределения памяти для многопоточных приложений и произведён выбор метода-прототипа. Предложен новый метод.

В конструкторском разделе приведены структура предлагаемого метода, используемые алгоритмы и структуры данных.

В технологическом разделе описаны требования к программному и аппаратному обеспечению платформы, описаны основные компоненты разработанного программного обеспечения, реализующего предложенный метод.

В исследовательском разделе описаны проведённые с помощью ПО, реализующего разработанный метод, исследования скорости распределения памяти и накладные расходы на ее выделение.

Предлагаемым направлением развития является расширение количества метаданных для аллокаторов для больших и малых объектов, избавление от примитивов синхронизации между потоками и разработка кэширования для данных локальных для потока для их дальнейшей передачи другим потокам.
\end{abstract}
