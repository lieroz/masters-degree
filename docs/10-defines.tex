\Defines % Необходимые определения. Вряд ли понадобться
\begin{description}
	%\item[NAME] Description \cite{def/Book} example
	\item[Аллокатор] англ. Allocator, программа, реализующая функции для управления памятью приложения.
	\item[Аллокация] англ. Allocation, процесс выделения памяти аллокатором приложению.
	\item[Куча] англ. Pageheap, структура данных, с помощью которой реализована динамически распределяемая память приложения, а также объём памяти, зарезервированный под эту структуру.
	\item[Арена] англ. Arena, область в куче процесса, которая организует управление памятью множеста страниц памяти.
	\item[Ячейка] англ. Bin, структура данных описывающая блок памяти в списке свободных блоков.
	\item[Страница памяти] англ. Page, блок фиксированного размера, на которые делится как физическая, так и виртуальная память, обычно равная размеру в 4КБ.
	\item[Большая страница] англ. Hugepage, блок фиксированного размера, на которые делится как физическая, так и виртуальная память, может иметь разный размер на разных архитертурах, на x86 ее размер равен 2МБ.
	\item[Гиперпоток] англ. Hyperthread, 
	\item[Спан] англ. Span, логическая структура, которая является совокупностью из нескольких системных страниц.
	\item[Буффер ассоциативной трансляции] англ. TLB (Translation lookaside buffer), специализированный кэш центрального процессора, используемый для ускорения трансляции адреса виртуальной памяти в адрес физической памяти.
	\item[Слаб] англ. Slab, представляет собой непрерывный фрагмент памяти, обычно состоящий из нескольких физически смежных страниц, фактический контейнер данных, связанных с объектами определенного типа содержащего кэша.
\end{description}

%%% Local Variables:
%%% mode: latex
%%% TeX-master: "rpz"
%%% End:
