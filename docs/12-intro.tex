\Introduction

Управление памятью является одной из ключевых задач операционных систем на сегодняшний день. Сама оперативная память, по сравнению с тем что было 30 лет назад,  стоит достаточно  дешего и ее объемы в серверах и персональных компьютерах   увеличились с нескольких килобайт до сотен гигабайт. Но сам факт   того, что памяти стало больше, не означитает ее бесконечность. Поэтому проводятся исследования на тему того, как лучше и эффективней управлять памятью   сегодня. Есть еще один немаловажный факт, который оказывает  сильное влияние на написание эффективного аллокатора памяти, - использование потоков в приложениях. Если еще 10 лет назад все что унас было - это 4 физических ядра в ПК или 8 физических ядер в серверах,то сейчас на серверах могут быть сотни процессорных ядер, тогда как у персональных компьютеров десятки. Сразу становится ясно, использовать старые подходы к управлению памятью не получится и надо их улучшать и исследовать новые способы.

В данной работе предлагается новый метод управления памятью для многопоточных приложений.