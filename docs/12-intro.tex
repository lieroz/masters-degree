\Introduction

Управление памятью является одной из ключевых задач операционных систем, а сама память была и будет одним из основных ресурсов. Несмотря на значительный рост объема физической памяти, из-за постоянного увеличения потребностей приложений, она остается дефицитным ресурсом. Важнейшим фактором, влияющим на эффективность распределения памяти является многопоточность приложений и особенности самих потоков, которые не имеют собственного адресного пространства. Как известно, владельцем ресурсов в ОС является процесс несмотря на то, что ресурсы выделяются и освобождаются при работе потоков. Потоки запрашивают динамическую память и могут завершаться без ее освобождения. Причем память запрашивается, например,  для размещения структур, массивов или создания иных структур данных в виртуальном адресном пространстве процесса. А физическая память при этом выделяется страницами по 4 КБ. И такое выделение является очень затратным с точки зрения использования памяти как основного ресурса системы.\cite{os-concepts}

В данной работе предлагается новый метод управления памятью для многопоточных приложений.